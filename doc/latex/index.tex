\begin{quotation}
\section*{E\-E\-C\-S 678 -\/ Project 3 -\/ The Buddy Allocator}



\end{quotation}


\subsection*{Introduction}

In this assignment, you will implement the buddy allocator. Much of the implementation has already been done for you. Your task is to implement functions that marked as T\-O\-D\-O in the \hyperlink{buddy_8c}{buddy.\-c}.

\subsection*{Installation}

To only build the buddy allocator use\-: \begin{quotation}
{\ttfamily \$ make}

\end{quotation}


To generate this documentation in H\-T\-M\-L use\-:

\begin{quotation}
{\ttfamily \$ make doc}

\end{quotation}


To clean the project use\-: \begin{quotation}
{\ttfamily \$ make clean}

\end{quotation}


\subsection*{Usage}

To run the executable use\-: \begin{quotation}
{\ttfamily \$ ./buddy $<$ test-\/files/test\-\_\-sample1.\-txt}

\end{quotation}
or \begin{quotation}
{\ttfamily \$ ./buddy -\/i test-\/files/test\-\_\-sample1.\-txt}

\end{quotation}


\subsection*{Testing}

Be sure you thoroughly test your program. We will use different test files than the ones provided to you. We have provided a simple test case to demonstrate how to write the test cases. We have also provided a testing script {\ttfamily runtests.\-sh} which will run all of your test files in the test-\/files directory. To execute all of your tests in that directory, simply use either\-:

\begin{quotation}
{\ttfamily \$ make test}

\end{quotation}
or \begin{quotation}
{\ttfamily \$ ./run\-\_\-tests.sh}

\end{quotation}


All test files must be located in the test-\/files directory and have the prefix \char`\"{}test\-\_\-\char`\"{} (i.\-e. test\-\_\-sample2.\-txt). The file test\-\_\-sample2.\-txt has the following lines in it\-:

\begin{quotation}
{\ttfamily a = alloc(44\-K)} \par
 {\ttfamily free(a)}

\end{quotation}


This test case allocates a 64 kilo-\/byte block of memory and assigns it to the variable 'a'. If the 'K' in the size argument is removed, then this call will only request 44 bytes. This test case then releases the block that is assigned to 'a' with the free command. Variable names can only be one character long, alphabetic letters.

Output must match exactly for credit. We have provided some sample output from our implementation in the test-\/files directory. All files that you wish to compare tests against should be located in the test-\/files directory and must match the name of its corresponding test file with the prefix \char`\"{}result\-\_\-\char`\"{} instead of \char`\"{}test\-\_\-\char`\"{}. Nothing you add to the code should print to standard output by the time you submit the project. 